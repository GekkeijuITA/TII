\documentclass[12pt]{article}
\usepackage[italian]{babel}
\usepackage{geometry}
\usepackage{amsmath}
\usepackage{graphicx}
\usepackage{amssymb}
\usepackage{array}

\geometry{margin=2cm}

\title{Formulario A.A 2023/2024}
\author{}

\begin{document}
\maketitle
\section{Esercizio 1}
\begin{itemize}
    \item \textbf{Anagrammi}: $A = \frac{\text{totale lettere}!}{\text{lettere uguali}!\cdot\text{altre lettere uguali}!}$
    \item \textbf{Combinazione senza duplicati}: $C = n\cdot (n-1)\cdot\ldots\cdot(n-k)$
    \item \textbf{Combinazione con duplicati}: $C = n!$
    \item \textbf{Gruppi di amici}: $G = \frac{\text{totale}!}{\text{cardinalità gruppo}!\cdot(\text{totale-cardinalità gruppo})}$
\end{itemize}
\section{Esercizio 2}
\begin{itemize}
    \item \textbf{Valore atteso}: $E[g(X)] = \sum_{i=1}^{n}g(x_i)\cdot p_i$
    \item \textbf{Valore atteso con la variabile che fa parte di una congiunta}: $E[g(X)] = \sum_{y}\sum_{x}=g(X=x)\cdot p(x,y)$
    \item \textbf{Probabilità condizionata}: $P(X|Y) = \frac{P(X,Y)}{P(Y)}$
    \item \textbf{Trovare probabilità (marginali) di una variabile che fa parte di una congiunta}: $P(X) = \sum_{y}P(X,y)$
\end{itemize}
\section{Esercizio 3}
\subsection{Funzione densità di probabilità (pdf)}
\begin{equation*}
    f(x) = \begin{cases}
        a & \text{se } x \in [y,z]\cup[w,k]\\
        0 & \text{altrimenti}
    \end{cases}
\end{equation*}
\begin{equation*}
    a = \int_{y}^{z}a \,dx + \int_{w}^{k}a \,dx = 1
\end{equation*}
\subsection{Funzione di distribuzione cumulativa (cdf)}
\begin{equation*}
    F(X) = \begin{cases}
        0 & \text{se } x < y\\
        \int_{y}^{x}a \,dx & \text{se } x \in [y,z]\\
        \int_{w}^{x}a \,dx  & \text{se } x \in [w,k]\\
        1 & \text{se } x > k
    \end{cases}
\end{equation*}
\section{Esercizio 4}
Domanda teorica
\section{Esercizio 5}
\begin{itemize}
    \item Bit di guadagno se equiprobabili: $\log_{2}(\text{\#possibilità})-\log_{2}(\text{\#tentativi})$
    \item Bit di guadagno se non equiprobabili: $\log_{2}(\frac{1}{\frac{1}{\text{\#possibilità}}})-\log_{2}(\frac{1}{\frac{\text{\#tentativi}}{\text{\#possibilità}}})$
    \item Entropia condizionata: $H(X|Y)=H(X,Y)-H(Y)$, $H(Y|X)=H(X,Y)-H(X)$. Se $X$ e $Y$ sono indipendenti, $H(X|Y)=H(X)$ altrimenti 0.
    \item Entropia congiunta: $H(X,Y) = H(Y)+H(X|Y)=H(X)+H(Y|X)$
\end{itemize}
\section{Esercizio 6}
\begin{itemize}
    \item Trovare codifica convoluzionale per $x[k]$ date le equazioni di parità $y_{i}[n]$: sostituisci $n$ con $k$ e risolvi le operazioni eseguendo poi il modulo 2.
    \item Entropia già spiegata nell'es 5
\end{itemize}
\section{Esercizio 7}
\begin{itemize}
    \item Determinare se è codifica di Huffman/istantanea: $\sum_{x\in\text{alfabeto}}2^{-L_{C}(x)}\leq 1$
    \item Lunghezza attesa: $H(X)=\sum_{i=1}^{|\text{alfabeto}|}p_{i}\log_{2}\frac{1}{p_{i}}$
    \item Determinare se è ottimale: \begin{enumerate}
        \item Calcolo lunghezza attesa
        \item $L(C,\text{alfabeto})=\sum_{i=1}^{|\text{alfabeto}|} p_{i}\cdot L_{C}(x_{i})$
        \item Se $L(C,\text{alfabeto}) = H(X)$ allora è ottimale, $\leq H(X)$ c'è spreco, $> H(X)$ c'è inefficienza.
    \end{enumerate}
\end{itemize}
\section{Esercizio 8}
\begin{itemize}
    \item Codifica aritmetica
    \item Codifica di Huffman a blocchi: \begin{enumerate}
        \item Determinare tutti i possibili blocchi
        \item Calcolarne la probabilità moltiplicando le probabilità dei singoli simboli tra di loro
        \item Applicare Huffman
    \end{enumerate}
\end{itemize}
\section{Esercizio 9/10}
\begin{itemize}
    \item Cassetto di monete: \begin{enumerate}
        \item Determino probabilità di ottenere le monete $P(M_{k})$
        \item Determino probabilità di ottenere testa/croce con le monete $P(x|M_{k})$
        \item Calcolare probabilità di ottenere testa/croce pescando a caso: $P(x) = \sum_{k}P(M_{k})\cdot P(x|M_{k})$
        \item Calcolare probabilità di ottenere testa/croce usando la stessa moneta: $P(xy) = P(x)-(1-P(x))$
    \end{enumerate}
    \item Verosomiglianza di una sequenza: $L(x|\theta)=\Pi_{i}p_{i}$
    \item Verosomiglianza distribuita uniformemente: $L(x|S)=\begin{cases}
        \frac{1}{S^{n}} & \text{se } S \geq x_{max}\\
        0 & \text{se } S < x_{max}
    \end{cases}$, nel caso dei taxi $x_{max}$ è il numero massimo delle licenze di conseguenza metteremo che $S=x_{max}$ se non vengono fornite le ipotesi altrimenti $S$ saranno le ipotesi e $n$ è il numero di taxi osservati.
\end{itemize}
\section{Esercizio 11}
\begin{itemize}
    \item Conferma che sia matrice di transizione: la somma delle righe deve essere 1
    \item Conferma che sia regolare: esiste un passo $t$ tale che $P^{t}_{rc}>0$
    \item Distribuzione stazionaria:\begin{enumerate}
        \item $\pi = \pi\cdot P$ con $\pi_{i}$ somma degli elementi della colonna
        \item Condizione: $\sum_{i}\pi_{i}=1$
        \item Isolo una delle $\pi_{i}$ e risolvo il sistema
        \item $\pi = \begin{pmatrix}
            \pi_{1} & \pi_{2} & \ldots & \pi_{n}
        \end{pmatrix}$
    \end{enumerate} 
    \item Distribuzione limite: $\lim_{t\to\infty}P^{t} = \lambda_{j}$ esiste se $P^{t}$ converge alla matrice con tutte le righe uguali a $\lambda_{j}$, non può esistere se la matrice non è regolare. Se la matrice è irriducibile e aperiodica, la distribuzione limite esiste e coincide con la distribuzione stazionaria.
\end{itemize}
\end{document}