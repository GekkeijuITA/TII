\documentclass[12pt]{article}
\usepackage[italian]{babel}
\usepackage{geometry}
\usepackage{amsmath}
\usepackage{graphicx}
\usepackage{amssymb}

\geometry{margin=2cm}

\title{Definizione assiomatica di probabilità}
\author{Lorenzo Vaccarecci}
\date{1 Marzo 2024}

\graphicspath{{./Immagini/}}
\newtheorem{example}{Esempio}

\begin{document}
\maketitle
\section{Nozioni fondamentali}
\begin{description}
    \item[Spazio campionario]: l'insieme \(S\) dei possibili risultati di un esperimento
    \item[Evento]: un qualunque sottoinsieme \(E\) di \(S\) che si \textit{realizza} se il risultato dell'esperimento appartiene a \(E\)  
\end{description}
Indichiamo con \(E \cup F\) l'unione degli eventi \(E\) e \(F\) e con \(EF\) la loro intersezione. \(EF=\emptyset\) sono mutuamente esclusivi. L'evento \(E^{c}\)(tutto ciò che non ha \(E\)) tale che \(E\cup E^{c}=S\) è il complementare di \(E\). 
\section{Assiomi}
\begin{description}
    \item[A1] \(0\leq P(E) \leq 1 \forall E \subseteq S\)
    \item[A2] \(P(S)=1\)
    \item[A3] Se gli eventi \(E_{i}\) con \(i=1,2,\dots\) sono mutuamente esclusivi, allora \(P(\cup_{i=1}^N E_{i})=\sum_{i = 1}^{N} P(E_{i})\)   
\end{description}
Se \(E \text{ e } F\) sono mutuamente esclusivi allora \(P(E \cup F)=P(E)+P(F)\).\\Se \(E \text{ e } F\) \textbf{non} sono mutuamente esclusivi allora \(P(E \cup F)=P(E)+P(F)-P(EF)\).
\section*{Proprietà}
\begin{itemize}
    \setlength\itemsep{0em}
    \item \(P(\emptyset)=0\)
    \item \(P(E^{C})=1-P(E)\)
    \item \(F\subseteq E \subseteq S \rightarrow P(F) \leq P(E)\)
\end{itemize}
\end{document}