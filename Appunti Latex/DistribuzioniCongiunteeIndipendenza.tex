\documentclass[12pt]{article}
\usepackage[italian]{babel}
\usepackage{geometry}
\usepackage{amsmath}
\usepackage{graphicx}
\usepackage{amssymb}
\usepackage{pgfplots}

\geometry{margin=2cm}

\title{Distribuzioni Congiunte e Indipendenza}
\author{Lorenzo Vaccarecci}
\date{4 Aprile 2024}

\graphicspath{{./Immagini/}}

\begin{document}
\maketitle
\section{Caso discreto}
La distribuzione di una coppia di variabili congiunte $(X,Y)$ con valori $\{x_{1},\ldots,x_{N}\}$ e $\{y_{1},\ldots,y_{M}\}$ è data da una \textit{pmf}
\begin{equation*}
    P(X= x_{i}, Y = y_{j}) = p(x_{i},y_{j}) \quad \text{ con } i = 1,2,\ldots,N \quad j = 1,2,\ldots,M
\end{equation*}
\begin{equation*}
    \sum_{i=1}^{N} \sum_{j=1}^{M} p(x_{i},y_{j}) = 1
\end{equation*}
\section{Distribuzioni marginali}
\subsection*{Storiella}
\begin{center}
    \(
        \begin{bmatrix}
            p(1,1) & p(1,2) & \ldots & p(1,M) \\ 
            p(2,1) & p(2,2) & \ldots & p(2,M) \\
            \vdots & \vdots & \ddots & \vdots \\
            p(N,1) & p(N,2) & \ldots & p(N,M)
        \end{bmatrix}    
    \)
\end{center}
$$
\left[\begin{array}{cccc}  
    (x_{1},y_{1}) & (x_{1},y_{2}) & \ldots & (x_{1},y_{M}) \\
    (x_{2},y_{1}) & (x_{2},y_{2}) & \ldots & (x_{2},y_{M}) \\
    \vdots & \vdots & \ddots & \vdots \\
    (x_{N},y_{1}) & (x_{N},y_{2}) & \ldots & (x_{N},y_{M})
\end{array}\right] \begin{array}{c}
	\sum_{j=1}^{M} p(1,j) \\
	\vphantom{} \\
    \vphantom{} \\
    \vphantom{}
\end{array}
$$
$$
\begin{array}{cccc}
    \sum_{i=1}^{N} p(i,1) & \hphantom{} & \hphantom{} & \hphantom{} 
\end{array}
$$
\begin{equation*}
    p_{X}(x_{i}) = P(X=x_{i}) = \sum_{j=1}^{M} p(x_{i},y_{j}) \quad \text{ e } \quad p_{Y}(y_{j}) = P(Y=y_{j}) = \sum_{i=1}^{N} p(x_{i},y_{j})
\end{equation*}
$p_{X}$ è praticamente la somma di una riga, mentre $p_{Y}$ è la somma di una colonna.
\begin{equation*}
    p_{X}+p_{Y} = 2
\end{equation*}
\end{document}