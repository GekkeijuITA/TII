\documentclass[12pt]{article}
\usepackage[italian]{babel}
\usepackage{geometry}
\usepackage{amsmath}
\usepackage{graphicx}
\usepackage{amssymb}
\usepackage{pgfplots}

\geometry{margin=2cm}

\title{Variabile Aleatoria}
\author{Lorenzo Vaccarecci}
\date{14 Marzo 2024}

\graphicspath{{./Immagini/}}

\begin{document}
\maketitle
\section{Variabili Casuali}
Una variabile casuale è una funzione a valori reali definita sullo spazio campionario.\\
\(X: S \rightarrow \{x_{1}, \dots, x_{n}\}\)
\subsection{Esempio}
$X$ somma del lancio di due dadi.
\section{Funzione di probabilità di massa}
Nel caso di una variabile $X$ a valori discreti $x_{i}$ con $i=1,2,\dots,n$, la funzione di probabilità di massa $p(\cdot)$ definita sulla retta reale, o \textit{pmf} o anche solo \textit{funzione di probabilità}, contiene tutta l'informazione necessaria per descrivere completamente $X$.\\ Si ha che \(p(x_{i})=P(X=x_{i})\geq 0 \text{ con } \sum_{i=1}^{n}p(x_{i})=1\).\\
\textit{La notazione corretta sarebbe $p(\{x_{i}\})$}
\subsection{Esempio continuato}
\(x_{1}=2,x_{2}=3,x_{3}=4,x_{4}=5,x_{5}=6,x_{6}=7,x_{7}=8,x_{8}=9,x_{9}=10,x_{10}=11,x_{11}=12\)\\
\(p_{1}=\frac{1}{36},p_{2}=\frac{2}{36},p_{3}=\frac{3}{36},p_{4}=\frac{4}{36},p_{5}=\frac{5}{36},p_{6}=\frac{6}{36},p_{7}=\frac{5}{36},p_{8}=\frac{4}{36},p_{9}=\frac{3}{36},p_{10}=\frac{2}{36},p_{11}=\frac{1}{36}\)
\section{Funzione di probabilità cumulata}
Ordiniamo i valori $x_{i}$ in modo tale che $x_{1}<x_{2}<\dots<x_{i}<\dots<x_{n}$ e introduciamo la funzione di probabilità cumulata $F(a)$, o \textit{cdf}, definita come
\begin{center}
    \(F(a)=\sum_{x_{i}\leq a}p(x_{i})\)
\end{center}
\begin{center}
    \begin{tikzpicture}
        \begin{axis}[xlabel=$x$, ylabel=$F(a)$, xmin=0, xmax=6,ymin=0,ymax=1,xtick={1,2,3,4,5,6},ytick={1/6,2/6,3/6,4/6,5/6,1},yticklabels={$\frac{1}{6}$,$\frac{2}{6}$,$\frac{3}{6}$,$\frac{4}{6}$,$\frac{5}{6}$,$1$}]
            \addplot+[const plot, black] coordinates {(1,1/6) (2,2/6) (3,3/6) (4,4/6) (5,5/6) (6,1)};
        \end{axis}
    \end{tikzpicture}
\end{center}
L'\textit{i}-esimo gradino è localizzato nel punto $x_{i}$ e il salto corrispondente vale $P(x_{i})$. La somma di tutti i gradini, ovviamente, è sempre 1.
\subsection{Valore atteso (o Speranza, Espettazione)}
Il valore atteso $\mu$ di una variabile casuale $X$ è indicato con $\mathbb{E}\left[X\right]$ ed è la media pesata dei valori $x_{i}$ che può assumere $X$. Ogni $x_{i}$ è pesato con la sua probabilità $p(x_{i})$ e quindi si ha
\begin{center}
    \(\mu=\mathbb{E}\left[X\right]=\sum_{i=1}^{n}x_{i}p_{i}\)
\end{center}
Il valore atteso di una variabile casuale non è casuale!
\section{Varianza}
Una seconda quantità che cattura proprietà importanti di una variabile casuale $X$ è la varianza $Var(X)$ definita come $Var(X)=E\left[(X-\mu)^{2}\right]$.\\
In parole povere è la dispersione dei valori intorno al valore medio.
\subsection{Deviazione standard}
Una quantità molto usata è la radice quadrata della varianza, nota come deviazione standard, o
\begin{center}
    \(SD(X)=\sqrt{Var(X)}\)
\end{center}
\section{Funzione di variabile aleatoria discreta}
$g:\{x_{1},\dots,x_{n}\} \rightarrow \mathbb{R}$
\begin{center}
    \(\mathbb{E}\left[g(x)\right]=\sum_{i=1}^{n}g(x_{i})p_{i}\)
\end{center}
\subsection{Esempio}
\(g(x)=5X+3 \rightarrow\\ \mathbb{E}\left[g(x)\right]=\sum_{i=1}^{n}(5x_{i}+3)p_{i}=\sum_{i=1}^{n}5x_{i}p_{i}+\sum_{i=1}^{n}3p_{i}=5\sum_{i=1}^{n}x_{i}p_{i}+3\sum_{i=1}^{n}p_{1}=5\mathbb{E}\left[X\right]+3\)
\end{document}