\documentclass[12pt]{article}
\usepackage[italian]{babel}
\usepackage{geometry}
\usepackage{amsmath}
\usepackage{graphicx}
\usepackage{amssymb}
\usepackage[dvipsnames]{xcolor}
\usepackage{tikz}

\geometry{margin=2cm}

\title{Probabilità Condizionata}
\author{Lorenzo Vaccarecci}
\date{8 Marzo 2024}

\graphicspath{{./Immagini/}}

\begin{document}
\maketitle
\section{Esercizio}
\textit{Date due monete A (70\% testa) e B (40\% testa), con quale probabilità prendi la moneta A ed esce testa?}\\
\textbf{Soluzione: }\(P(AT)=P(T|A)P(A)=0.7\cdot\frac{1}{2}=0.35\)
\section{Esercizio}
\textit{Dato un mazzo da 52 carte, calcolare la probabilità di fare 4 mazzi da 13 carte con un asso}
\begin{itemize}
    \item $E_{1}$: picche 1 mazzo
    \item $E_{2}$: $E_{1}$ e cuori in un altro
    \item $E_{3}$: $E_{1}$ e $E_{2}$ e quadro in un altro
    \item $E_{4}$: $E_{1}$ e $E_{2}$ e $E_{3}$ e fiori in un altro
\end{itemize}
\textbf{Soluzione: }\(P(E_{1}E_{2}E_{3}E_{4})=P(E_{1})P(E_{2}|E_{1})P(E_{3}|E_{2}E_{1})P(E_{4}|E_{3}E_{2}E_{1}) = 1\cdot \frac{39}{51} \cdot \frac{26}{50} \cdot \frac{13}{49}\)
\section{Esercizio}
\textit{Con 8 palline Rosse e 4 Bianche, qual è la probabilità di avere estratto due palline Rosse ($R_{1}R_{2}$)?}\\
\textbf{Soluzione: }\(P(R_{1}R_{2})=P(R_{2}|R_{1})P(R_{1})=\frac{7}{11}\cdot \frac{8}{12}=\frac{14}{33}=\frac{\binom{8}{2}}{\binom{12}{2}}\)
\section{Esercizio}
\textit{Calcolare la probabilità con cui una persona potrebbe essere positiva a una patologia se: la sensibilità del test è del 95\% (probabilità di dire Positivo se Malato), specificità del 99\% (probabilità di dire Negativo se Sano) e che l'incidenza della malattia è del 0.2\%}\\
\textbf{Soluzione: }\(P(M|P)=\frac{P(P|M)P(M)}{P(P|M)P(M)+P(P|S)P(S)}=\frac{0.95\cdot 0.02}{0.95\cdot 0.02 + 0.01\cdot 0.998}=0.016\)
\section{Esercizio}
\textit{Abbiamo 3 carte A(\textcolor{red}{RR}),B(\textcolor{red}{R}B),C(BB). Butto una carta e esce \textcolor{red}{R}, qual è la probabilità che anche l'altra faccia sia \textcolor{red}{R}? (oppure qual è la probabilità che la cartia buttata sia la A)}\\
\(P(A|R_{1})=\frac{\textcolor{Orchid}{P(R_{1}|A)}P(A)}{P(R_{1}|A)P(A)+\textcolor{Cyan}{P(R_{1}|B)}P(B)+\textcolor{red}{P(R_{1}|C)}P(C)}=\frac{1\cdot \frac{1}{3}}{1\cdot \frac{1}{3}+\frac{1}{2}\cdot \frac{1}{3}+0\cdot\frac{1}{3}}=\frac{2}{3}\)
\begin{description}
    \item[] \textcolor{Orchid}{*Ci sono due facce rosse} 
    \item[] \textcolor{Cyan}{*C'è una faccia rossa}
    \item[] \textcolor{red}{*Non ci sono facce rosse}
\end{description}
\section{Esercizio}
\begin{center}
    \begin{tikzpicture}
        \draw[draw=black] (11.1,5.5) rectangle ++(0.9,2) node[pos=.5]{A};
        \draw[draw=black] (13.1,5.5) rectangle ++(0.9,2) node[pos=.5]{B};
        \draw[draw=black] (15.1,5.5) rectangle ++(0.9,2) node[pos=.5]{C};
    \end{tikzpicture}
\end{center}
\textit{Scelgo A. Qual è la probabilità che dietro alla porta A ci sia la macchina se so che nella porta C c'è la capra ($R_{c}$)?}\\
\(P(A|R_{c})=\frac{\textcolor{Orchid}{P(R_{c}|A)}P(A)}{\textcolor{Orchid}{P(R_{c}|A)}P(A)+\textcolor{Cyan}{P(R_{c}|B)}P(B)+\textcolor{red}{P(R_{c}|C)}P(C)}=\frac{\frac{1}{2}\cdot\frac{1}{3}}{\frac{1}{2}\cdot\frac{1}{3}+1\cdot\frac{1}{3}+0\cdot\frac{1}{3}}=\frac{1}{3}\)
\begin{description}
    \item[] \textcolor{Orchid}{*Se apro la porta $A$ il gioco finisce perchè è quella che ha scelto il concorrente} 
    \item[] \textcolor{Cyan}{*Se il concorrente sceglie $A$, io presentatore apro la porta dove non c'è la macchina quindi apro $C$}
\end{description}
\textit{Quindi qual è la probabilità che la macchina sia dietro alla porta B?}\\
\(P(A|R_{c})=\frac{\textcolor{Orchid}{P(R_{c}|B)}P(B)}{\textcolor{Orchid}{P(R_{c}|A)}P(A)+\textcolor{Cyan}{P(R_{c}|B)}P(B)+\textcolor{red}{P(R_{c}|C)}P(C)}=\frac{1\cdot\frac{1}{3}}{\frac{1}{2}\cdot\frac{1}{3}+1\cdot\frac{1}{3}+0\cdot\frac{1}{3}}=\frac{2}{3}\)
\end{document}