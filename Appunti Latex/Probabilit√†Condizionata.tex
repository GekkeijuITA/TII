\documentclass[12pt]{article}
\usepackage[italian]{babel}
\usepackage{geometry}
\usepackage{amsmath}
\usepackage{graphicx}
\usepackage{amssymb}

\geometry{margin=2cm}

\title{Probabilità Condizionata}
\author{Lorenzo Vaccarecci}
\date{7 Marzo 2024}

\graphicspath{{./Immagini/}}
\newtheorem{example}{Esempio}

\begin{document}
\maketitle
In generale quando, dati due eventi $E$ e $F$, siamo interessati a calcolare la probabilità di $E$ quando sappiamo che si è realizzato $F$. La probabilità di $E$ condizionata a $F$, indicata come $P(E|F)$, è definita come
\begin{equation*}
    P(E|F)=\frac{P(EF)}{P(F)}
\end{equation*}
\begin{description}
    \item[Osservazione 1] Lo spazio campionario si è ridotto da $S$ a $F$ e misura $P(F)$
    \item[Osservazione 2] Quando non c'è intersezione, la probabilità è nulla
    \item[Osservazione 3] \(P(E|F)=\frac{P((A\cup A^{C})F)}{P(F)}=\frac{P(AF)}{P(F)}+\frac{P(A^{C}F)}{P(F)}=P(A|F)+P(A^{C}|F)\)  
\end{description}
La probabilità condizionata soddisfa gli assiomi delle probabilità con $F$ come spazio campionario.
\section{Regola della moltiplicazione}
\textbf{Base: } \begin{equation*}
    P(ABC)=P(AB|C)P(C)=P(A|BC)P(B|C)P(C)
\end{equation*}
\textbf{Generale: } \begin{equation*}
    P(E_{1}E_{2}\dots E_{n})=P(E_{1})P(E_{2}|E_{1})P(E_{3}|E_{1}E_{2})\dots P(E_{n}|E_{1}E_{2}\dots E_{n-1})
\end{equation*}
\section{Teorema di Bayes}
Per ogni coppia di eventi $E$ e $F$, applicando la definizione di probabilità condizionata, possiamo riscrivere $P(F|E)$ in termini di $P(E|F)$ (o viceversa). Ovvero
\begin{equation*}
    P(E|F) = \frac{P(E|F)P(E)}{P(F)}
\end{equation*}
\subsection*{Formula della probabilità assouluta}
\begin{equation*}
    P(E)=P(EF)+P(EF^{C})=P(F)P(E|F)+P(F^{C})P(E|F^{C})
\end{equation*}
Quindi il teorema di Bayes diventa:
\begin{equation*}
    P(F|E) = \frac{P(E|F)P(F)}{P(F)P(E|F)+P(F^{C})P(E|F^{C}}
\end{equation*}
\section{Eventi indipendenti}
\begin{equation*}
    P(EF)=P(E)\cdot P(F) \rightarrow P(E|F)=\frac{P(E)P(F)}{P(F)}=P(E)
\end{equation*}
\end{document}